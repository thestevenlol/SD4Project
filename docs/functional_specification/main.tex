\documentclass[a4paper,12pt]{article}
\usepackage[utf8]{inputenc}
\usepackage{graphicx}
\usepackage[hidelinks]{hyperref}
\usepackage{amsmath}
\usepackage{amssymb}
\usepackage{geometry}
\geometry{margin=1in}

\title{Functional Specification}
\author{Your Name}
\date{\today}

\begin{document}

\maketitle

\tableofcontents

\section{Introduction}
\subsection{Purpose}
Describe the purpose of the document.

\subsection{Scope}
Describe the scope of the project.

\subsection{Definitions, Acronyms, and Abbreviations}
List and define any terms, acronyms, and abbreviations used in the document.

\subsection{References}
List any references used in the document.

\subsection{Overview}
Provide an overview of the document structure.

\section{Overall Description}
\subsection{Product Perspective}
Describe the context and origin of the product.

\subsection{Product Functions}
List the major functions the product will perform.

\subsection{User Characteristics}
Describe the characteristics of the intended users.

\subsection{Constraints}
List any constraints that will affect the design and implementation.

\subsection{Assumptions and Dependencies}
List any assumptions and dependencies.

\section{Specific Requirements}
\subsection{External Interface Requirements}
Describe all external interfaces.

\subsection{Functional Requirements}
Detail the functional requirements.

\subsection{Performance Requirements}
Specify performance requirements.

\subsection{Design Constraints}
List any design constraints.

\subsection{Software System Attributes}
Describe attributes such as reliability, availability, security, etc.

\section{Appendices}
Include any additional information in appendices.

\end{document}